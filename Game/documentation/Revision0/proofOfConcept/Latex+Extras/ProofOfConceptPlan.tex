\documentclass[11pt, oneside]{article}   	% use "amsart" instead of "article" for AMSLaTeX format
\usepackage{geometry}                		% See geometry.pdf to learn the layout options. There are lots.
\geometry{letterpaper}                   		% ... or a4paper or a5paper or ... 
%\geometry{landscape}                		% Activate for rotated page geometry
%\usepackage[parfill]{parskip}    		% Activate to begin paragraphs with an empty line rather than an indent
\usepackage{graphicx}				% Use pdf, png, jpg, or eps§ with pdflatex; use eps in DVI mode
								% TeX will automatically convert eps --> pdf in pdflatex		
\usepackage{amssymb}

\usepackage{fancyhdr}
\pagestyle{fancy}
\fancyhead[L]{Gill, Surinder
		1308896\\
		Hu, Joshua
		1311940\\
		Lago, Nick
		1302613}
\fancyhead[C]{Proof of Concept Plan}
\fancyhead[R]{\today}

%SetFonts

%SetFonts

\title{Proof of Concept Plan}
\author{Gill, Surinder\\
		1308896
		\and
		Hu, Joshua\\
		1311940
		\and
		Lago, Nick\\
		1302613}
%\date{}							% Activate to display a given date or no date

\begin{document}

\maketitle
\thispagestyle{empty}

\newpage
\marginpar{}
\section*{Revision History}

\begin{table}[hp]
\caption{Revision History: Proof of Concept Plan}
\begin{center}
\label{tab:}
\begin{tabular}{|c|c|c|c|}
\hline
\textbf{DATE} & \textbf{DEVELOPER} & \textbf{CHANGE} & \textbf{REVISION}\\
\hline
September 30, 2015 & Gill, Surinder & Text & 1\\
\hline
September 30, 2015 & Lago, Nick & Text & 1\\
\hline
September 30, 2015 & Hu, Joshua & Typesetting & 1\\
\hline
\end{tabular}
\end{center}
\label{default}
\end{table}%


\newpage

\section*{Challenges \& Risks}
\subsection*{Most Significant Risk}

For our game reimplementation, there are no significant risks other than the hosting server crashing or having too many users accessing the website with the JavaScript game.

\subsection*{Will a part of the implementation be difficult?}

The part of the implementation that we?ll find most difficult is organizing the contents and source files of the game appropriately since they are all in one giant folder. Additionally, optimizing the code will be challenging since it was written by someone else without many comments. 

\subsection*{Will testing be difficult?}

Testing will not be difficult because we have worked with Java and JavaScript unit testing frameworks such as JUnit and QUnit. As a result, we?ll be able to use black box testing and single unit testing as we?ve learned previously in classes. 

\subsection*{Is a required library difficult to install?}

No, the required library is not difficult to install. The open source project is already implanted in our git repository with all its libraries. 

\subsection*{Will portability be a concern?}

No, portability will not be a concern since this is a browser-based web app. It will be able to be written on any operating software and also be accessed through any platform that has a web browser.

\end{document}  